\documentclass[a4paper, 12pt, titlepage]{scrartcl}  
\usepackage[utf8]{inputenc}
\usepackage[ngerman]{babel}

\pagenumbering{arabic} 
\usepackage{graphicx}
\usepackage[onehalfspacing]{setspace}
\usepackage[left=3cm,right=3cm,top=2cm,bottom=2cm,includeheadfoot]{geometry}

\begin{document}
\author{Supercoole Kinder}
\title{Projektdokumentation} 
\publishers{Humboldt-Universit\"at zu Berlin}
\maketitle
\tableofcontents

\section{Projektdokumentation}
	\subsection{Projektbeschreibung} % Ziele und Motivation, warum das alles so toll ist. (umweltfreundlich, super Auslastung, weniger Unfälle... besonderes feature: car2car kommunikation)
	\subsection{Lastenheft} %(nicht-)funktionale anforderungen, was liefern wir am ende, use cases. nur grob erwähnen und dann im anhang auf das lastenheft verweisen??
	\subsection{Projekt-Organisation und Umfeld} % 3 teams arbeiten parallel, zusammenarbeit mit assystem und frauenhofer, wir: agiler Ansatz, clion, gitlab, slack, meistertask; späterer Umschwung auf gitlab issues
	% ros, turtle sim
	% externer Bau der autos plus betreuung durch externen studenten/frauenhofer/assystems
	% Räumlichkeiten: wo fand was statt
	\subsection{Projektplan} % vgl. project charter http://sce2.umkc.edu/BIT/burrise/pl/appendix/Software_Documentation_Templates/Project_Charter_Template.html
	%Meilensteine, Routinen, Treffen, Planänderungen (mal gucken was die anderen Gruppe da haben), Designänderungen der Architektur nach treffen mit assystem?
	\subsection{Projektablauf} %vergleich ist-zustand mit soll-zustand? was wurde bei den wöchentlichen treffen mit schlingloff etc besprochen, Lieferprobleme...
	% eventuell sind Projektmetriken interessant: Diagram zur Visualsierung der Zeitverteilung auf verschiedene Aspekte des Projekts (wie viel Zeit für's Coden, in ros einarbeiten etc)
\section{Produktdokumentation}
	\subsection{Systemvoraussetzungen} % ubuntu, SW von assystems, ros. im prinzip beschreibung unserer vm + welche Schritte wurden unternommen ums zum laufen zu bringen und kurze erläuterung dass sw von assystems aufs board geschmissen wurde. und catkin zeugs
	\subsection{Fahrzeug-Architektur}
		\subsubsection{Hardware} % sensoren, boards, teile ausm 3d drucker, datenblätter im anhang? erläuterung von joe(?) wie er die boards zum laufen gebracht hat
		\subsubsection{Fahrzeugzusammenbau} % tutorial zum zusammenbau, fotos
	\subsection{Software-Architektur}
		\subsubsection{Architekturübersicht} % grafische Darstellung der Beziehung der module zueinander, uml?
		\subsubsection{Coding-Standard} % C++, Mix aus Google Style Guide und anderen Conventions, version control
		\subsubsection{Modulebeschreibung} % Beschreibung der Nodes und diagramm 
	\subsection{Tests}
		\subsubsection{Test-Standard} % warum haben wir uns auf google test festgelegt; Designentscheidung
		\subsubsection{Test-Cases}
		\subsubsection{Testergebnisse}
	\subsection{Nutzerhandbuch} %brumm brumm auto fahren (wie lasse ich=schlingloff die autos fahren/kann ich die simulation starten? was kann ich alles anstellen und wie? steuerung, hindernisse...)
\section{Beschreibung der Eigenleistung} % alphabetisch nach Namen
	\subsection{Cooles Kind 1}
	\subsection{Cooles Kind 2}
%Anhang: 
%Datenblätter
%Code
%Spezifikationen aus dem Orga Repo
%
\end{document}
