\documentclass[a4paper, 12pt, titlepage]{scrartcl}  
%\usepackage[T1]{fontenc}
\usepackage[utf8]{inputenc}
\usepackage[ngerman]{babel}

\usepackage{enumitem}
\setitemize{itemsep=0pt}

\begin{document}
\title{Risikoanalyse\\zum Projekt\\Hochautomatisiertes Fahren} 
\publishers{Humboldt-Universit\"at zu Berlin}
\maketitle
\tableofcontents
\newpage

\section{Akzeptanzkriterien}
\textbf{Beschreibung}: Erfolg des Projekts ist bedingt durch die erfüllung der Demo Akzeptanzkritierien. Diese sind:
		\begin{itemize}
			\item[1.] Ausgangssituation: 3 Fahrzeuge stehen hintereinander.
			\item[2.]
			\begin{itemize}
				\item[a)] Kunde stellt Hindernis in gerader Linie
				\item[b)] Kunde startet das komplette Szenario.
			\end{itemize}
			\item[3.] Die Fahrzeuge fahren im festem Abstand gerade aus und fahren in Kolonne bis zum Hindernis und bleiben dort stehen.
		\end{itemize}

	\subsection{Szenario läuft nicht fehlerlos ab}
		\textbf{Beschreibung}: Beliebiger unbekannter Fehler\\
		\textbf{Auswirkung}: Akzeptanzkriterium nicht erfüllt\\
		\textbf{Tragweite}: Projektgefährdend\\
		\textbf{Prävention}:
		\begin{itemize}
			\item Test unter Szenariobedingungen mit maximaler verfügbarer Redundanz
		\end{itemize}
		\textbf{Alternativen}: 

	\subsection{Szenario schlägt fehl wegen Bauteilfehler}
	\textbf{Beschreibung}: Unentdeckter Bauteilfehler bei Szenarioausführung\\
	\textbf{Auswirkung}: Akzeptanzkriterium nicht erfüllt\\
	\textbf{Tragweite}: Projektgefährdend\\
	\textbf{Prävention}: 
	\begin{itemize}
		\item Eine Woche vor Demo sind alle VIER Fahrzeuge im fahrbereiten Zustand
		\item maximal DREI werden in dieser Woche zum testen eingesetzt
		\item HW Bauanleitung ist komplett 1 Woche vorher vorhanden
	\end{itemize}
	\textbf{Alternativen}: 

	\subsection{Steuerung funktioniert nicht}
		\textbf{Beschreibung}: Steuerrechner bei Szenariostart funktioniert nicht\\
		\textbf{Auswirkung}: Akzeptanzkriterium nicht erfüllt\\
		\textbf{Tragweite}: Projektgefährdend\\
		\textbf{Prävention}:		
		\begin{itemize}
			\item Jede Gruppe hält einen Steuerrechner lauffähig
			\item Generalprobe mit allen drei Steuerrechnern
		\end{itemize}
		\textbf{Alternativen}: 

	\subsection{Bodenbeschaffenheit verändert Fahrverhalten}
		\textbf{Beschreibung}: Demoboden verändert fahrverhaltend der Fahrzeuge so, dass Kolonnenverhalten fehlschlägt oder es zur Kollision mit Hindernis kommt\\
		\textbf{Auswirkung}: Akzeptanzkriterium nicht erfüllt\\
		\textbf{Tragweite}: Projektgefährdend\\
		\textbf{Prävention}:
		\begin{itemize}
			\item Generalprobe vor Ort
		\end{itemize}
		\textbf{Alternativen}: 

	\subsection{Kamera erkennt Hinderniss falsch}
		\textbf{Beschreibung}: Passanten, Objekte neben Fahrbahn verändern Erkennungsverhalten. Kolonne hält nicht vor dem vom Kunden aufgestellten Objekt
		\textbf{Auswirkung}: Akzeptanzkriterium nicht erfüllt\\
		\textbf{Tragweite}: Projektgefährdend\\
		\textbf{Prävention}: \\
		\begin{itemize}
			\item Test mit Passanten und Objekten neben Fahrbahn
			\item eindeutige Markierung des Objektes
		\end{itemize}
		\textbf{Alternativen}: 

	
\newpage
\section{Technische Risiken}
	\subsection{Sach- und Personenschaden durch Kollision}
		\textbf{Beschreibung}: Sowohl in der Entwicklungsphase als auch bei der abschlie\ss enden Demonstration kann es dazu kommen, dass die Fahrzeuge mit anderen Objekten oder Menschen kollidieren. Eine Kollision kann wie folgt zustande kommen: 
			\begin{itemize}
				\item Ultraschallsensor/Kamera funktioniert nicht bzw. funktioniert nur unzureichend
				\item Daten von Ultraschallsensor/Kamera werden nicht schnell genug verarbeitet
				\item Bremse defekt bzw. funktioniert nur unzureichend
				\item Daten von anderen Fahrzeugen aus der Kolonne werden nicht (schnell genug) geliefert
				\item Supervisor\footnote{Person, welche \"uber Remote Controll Fahrzeug manuell steuern kann} erteilt Bremsbefehl nicht schnell genug
				\item Fehler in der Software f\"uhrte zur Kollision
			\end{itemize}
		\textbf{Auswirkung}: kann zu Besch\"adigung von Bauteilen der Fahrzeuge f\"uhren. Schwerwiegender Personenschaden unwahrscheinlich, da die Fahrzeuge mit niederigem Tempo fahren (<15km/h).\\
		\textbf{Tragweite}: nicht projektgef\"ahrdend, aber bei der Demo unbedingt zu vermeiden.\\
		Pr\"aventionsma\ss nahmen:
			\begin{itemize}
				\item physische Bremsen regelm\"a\ss ig pr\"ufen
				\item Umfangreiches Testen der Software
				\item Keine komplexeren Szenarien testen, bis Bremsfunktion befriedigend funktioniert
				\item Supervisor stets hochkonzentriert 
			\end{itemize}
		Alternativen: 
			\begin{itemize}
				\item alternative Bremsen/Sensoren von externer Quelle beziehen, wenn Funktionali\"at nur unzureichend
				\item bei Besch\"adigung: Ersatzteil beziehen vom Auftraggeber
			\end{itemize}
		
	\subsection{Sach- und Personenschaden durch Montage der Fahrzeuge}
		\textbf{Beschreibung}: Die Fahrzeuge werden manuell durch Teammitglieder zusammengebaut.\\
		\textbf{Auswirkung}: Personalausfall und ggf. Bedarf an Ersatzteilen.\\
		\textbf{Tragweite}: zu vernachl\"assigen\\
		Pr\"aventionsma\ss nahmen:
			\begin{itemize}
				\item Schulung der Teammitglieder, die mit dem Zusammenbau der Fahrzeuge betraut wurden
				\item Sichere, ruhige Arbeitsumgebung
				\item Beaufsichtigung durch Fachpersonal
			\end{itemize}
		Alternativen: 
			\begin{itemize}
				\item bei Personalausfall: anderes Teammitglied oder anderes Team mit dem Zusammenbau des Fahrzeugs beauftragen
				\item bei Besch\"adigung: Ersatzteil beziehen vom Auftraggeber
			\end{itemize}
		
	\subsection{Kommunikationsverlust der Fahrzeuge}
		\textbf{Beschreibung}: W\"ahrend eines Szenarios kann es passieren, dass es aus verschiedenen Gr\"unden dazu kommt, dass die Fahrzeuge untereinander bzw. mit dem Supervisor nicht mehr kommunizieren:
			\begin{itemize}
				\item WLAN-Dongle defekt
				\item Hohe Netzwerkauslastung
				\item Softwarefehler
				\item Signale k\"onnen nicht verarbeitet werden
			\end{itemize}
		\textbf{Auswirkung}: Kollisionen m\"oglich, diverse Szenarien der Ausbaustufen lassen sich ohne Kommunikation nicht durchf\"uhren.\\
		\textbf{Tragweite}: kritisch, da ohne echte Kommunikation unter den Fahrzeugen zwar das rudiment\"arste Akzeptanzkriterium erf\"ullt werden k\"onnte, es jedoch an der eigentlichen Auftragsbeschreibung vorbei zielt.\\
		Pr\"aventionsma\ss nahmen:
			\begin{itemize}
				\item Einigung auf einen Kommunikationsstandard zwischen den Teams (Kommunikationsprotokoll)
				\item Umfangreiches Testen der Software
				\item Entfernen von Objekten, welche die Kommunikation st\"oren k\"onnten  (z.B. Pflanzen)
			\end{itemize}
		Alternativen: 
			\begin{itemize}
				\item Analysieren des Netzwerkverkehrs durch geeignete Tools (z.B. Wireshark), um Probleme zu erkennen
				\item Testen mit alternativen Transportprotokoll (TCP oder UDP)
				\item Austausch WLAN-Dongle
			\end{itemize}
		
	\subsection{Ausfall der Hardware/Bauteile verschlei\ss en}
		\textbf{Beschreibung}: Hardware wurde bereits defekt geliefert oder stellt Funktionalit\"at w\"ahrend der Projektphase ein.\\
		\textbf{Auswirkung}: Fahrzeuge mit Hardwaredefekt/kaputten Bauteilen sind nicht funktionsf\"ahig.\\
		\textbf{Tragweite}: kritisch, da die Fahrzeuge im Mittelpunkt des Projekts stehen\\
		Pr\"aventionsma\ss nahmen:
			\begin{itemize}
				\item Pfleglicher Umgang mit den Fahrzeugen
				\item Begrenzte Geschwindigkeit der Fahrzeuge, um bei Kollisionen schwerwiegende Besch\"adigungen zu vermeiden
				\item bei manueller Steuerung durch Supervisor: defensive Fahrweise
			\end{itemize}
		Alternativen: 
			\begin{itemize}
				\item bei Besch\"adigung: Ersatzteil beziehen vom Auftraggeber
			\end{itemize}
		
	\subsection{Code von Assystem wird nicht p\"unktlich geliefert}
		\textbf{Beschreibung}: Die Firma Assystem hat den Teams in Aussicht gestellt, ihnen einen Spurhalteassistenten zur Verf\"ugung zu stellen. Jedoch wurde sich noch nicht auf ein konkretes Datum festgelegt. \\
		\textbf{Auswirkung}: Ohne einen Spurhalteassistenten sind die Ausbaustufen der Akzeptanzkriterien nicht umzusetzen.\\
		\textbf{Tragweite}: in Teilbereichen kritisch, da ohne Assistent s\"amtliche Funktionalit\"aten der Kolonnenfahrweise auf einer gerade Strecke demonstriert werden m\"ussten.\\
		Pr\"aventionsma\ss nahmen:
			\begin{itemize}
				\item regelm\"a\ss ige Kommunikation mit Assystem und Auftraggeber
			\end{itemize}
		Alternativen: 
			\begin{itemize}
				\item eigenen Spurhalteassistenten entwickeln (kritisch, da anderer thematischer Schwerpunkt des Projekts und sehr zeitaufw\"andig)
				\item alternative Quelle konsultieren (Benjamin Schlotter)
			\end{itemize}
		
	\subsection{Projekt kann nicht wie vereinbart fertiggestelt werden}
		\textbf{Beschreibung}: Die vom Kunden geforderten Akzeptanzkriterien waren umfangreicher als von beiden Seiten angenommen. Der zu Beginn aufgesetzte Vertrag konnte nicht in der vereinbarten Form erf\"ullt werden. Sowohl technische als auch organisatorische/soziale Probleme k\"onnen dazu f\"uhren, dass das Projekt nicht in der gew\"unschten Form umgesetzt werden kann.\\
		\textbf{Auswirkung}: Die Demonstration am Tag der Offenen T\"ur der HU (Mai 2018) kann nicht vorgef\"uhrt werden.\\
		\textbf{Tragweite}: projektgef\"ahrdend. 
		Pr\"aventionsma\ss nahmen:
			\begin{itemize}
				\item Einhalten der gesetzten Meilensteine
				\item regelm\"a\"ss ige Treffen innerhalb und zwischen den Teams, um zuk\"unftigem Arbeitsaufwand mit bereits Geleistetem besser einzusch\"atzen
				\item bei \"Uberforderung: Arbeitsaufteilung auf andere Teammitglieder/Teams fr\"uhstm\"oglich organisieren
				\item 
			\end{itemize}
		Alternativen: 
			\begin{itemize}
				\item Anfertigen einer Simulation anstatt Vorf\"uhren einer Live-Demo
			\end{itemize}
		
	\subsection{Datenverlust}
		\textbf{Beschreibung}: Datenverlust kann sowohl in der von der Universit\"at zur Verf\"ugung gestellten Infrastruktur(Server) als auch auf den Ger\"aten der einzelnen Teammitglieder eintreten:
			\begin{itemize}
				\item Stromausfall
				\item Hardwareausfall
				\item Diebstahl
				\item Datenkorruption
			\end{itemize}
		\textbf{Auswirkung}: Arbeitsprozesse werden in die L\"ange gezogen, Verlust an Effizienz.\\
		\textbf{Tragweite}: kritisch, da im schlimmsten Fall zu einem beliebigem Zeitpunkt X vor der endg\"ultigen Abgabe das gesamte Projekt von Vorne begonnen werden muss.\\
		Pr\"aventionsma\ss nahmen:
			\begin{itemize}
				\item Backups (auch auf externen Ger\"aten)
				\item Nutzen von Repositories(GitLab)
				\item Nutzen einer geeigneten Arbeitsumgebung (kontrollierter Zugang zum PC/Laptop, keine Fl\"ussigkeiten in der N\"ahe des PC/Laptops, Laptopschloss...)
			\end{itemize}
		Alternativen: 
			\begin{itemize}
				\item Bei Datenverlust auf Universit\"atsserver: Wenden an die RBG\footnote{Rechnerbetriebsgruppe}
			\end{itemize}

\newpage
\section{Organisatorische Risiken}
	\subsection{Schlechte Kommunikation innerhalb des Teams}
		\textbf{Beschreibung}: Dies betrifft u.a. missverst\"andliche Arbeitszuteilung, unklare Zust\"andigkeiten, Nichteinhalten von Terminen etc. Schlechte Kommunikation kann aber auch bedeuten, dass \"ueber zu viele Kommunikationskan\"ale kommuniziert werden soll und/oder einzelne Mitglieder des Teams die Granularit\"t der Kommunikation als viel zu fein einsch\"atzen, sodass die Kommunikation allein einen Overhead erzeugt, welcher zus\"atzlichen Arbeitsaufwand mit sich bringt.\\
		\textbf{Auswirkung}: die Frustration im Team steigt, es entstehen ineffiziente Arbeitsabl\"aufe, T\"atigkeiten werden redundant ausgef\"uhrt. Im schlimmsten Fall geben einzelne Teammitglieder oder ganze Teams das Projekt auf.\\
		\textbf{Tragweite}: projektgef\"ahrdend.\\
		Pr\"aventionsma\ss nahmen:
			\begin{itemize}
				\item Aufbau geeigneter Kommunikationskan\"ale
				\item Anzahl Kommunikationskan\"ale auf ein f\"ur das Projekt und das Team angemessenes Niveau halten
				\item regelm\"a\ss ige Treffen nicht als obligatorisch, sondern als verpflichtend einstufen
				\item jedem Treffen einer zuvor festgelegtem Thematik unterstellen
				\item Treffen moderieren (nicht im Sande verlaufen lassen und nicht in Details verlieren)
				\item Probleme direkt und m\"oglichst fr\"uh ansprechen, auch mit den Projektleitern
				\item mit Problem-Mitgliedern reden, ggf. auch \emph{offen} ank\"undigen, dass die Projektleitung im bestehenden Problemfall einbezogen wird
				\item Kritik sachlich \"uben
			\end{itemize}
		Alternativen: 
			\begin{itemize}
				\item Teamwechsel
				\item Wechsel des Teamleiters
			\end{itemize}

	\subsection{Schlechte Kommunikation zwischen den Teams}
		\textbf{Beschreibung}: Erkl\"artes Ziel in diesem Projekt ist es, dass die Fahrzeuge in festgelegten Szenarien direkt miteinander kommunizieren. Dazu ist es zwingend erforderlich, dass alle Teams auf der gleichen Basis arbeiten. Die Gr\"unde, weshalb hier die Kommunikation fehlschlagen k\"onnte, sind prinzipiell die gleichen wie bei der Kommunikation innerhalb eines Teams. Erschwerend kommt hinzu, dass neben der internen Terminfindung in diesem Fall auch noch externe Termine koordiniert werden m\"ussen.\\
		\textbf{Auswirkung}: erschwerte Entwicklung f\"ur das Kommunikationsprotokoll, Frustration, einzelne Arbeitsabschnitte werdne unn\"otig verl\"angert\\
		\textbf{Tragweite}: kritisch, da ohne echte Kommunikation unter den Fahrzeugen zwar das rudiment\"arste Akzeptanzkriterium erf\"ullt werden k\"onnte, es jedoch an der eigentlichen Auftragsbeschreibung vorbei zielt.\\
		Pr\"aventionsma\ss nahmen:
			\begin{itemize}
				\item Nicht zwangsweise komplette Teams zu Besprechungen anfordern, sondern Vertreter; ggf. Bildung von Expertengruppen
				\item Aufbau geeigneter Kommunikationskan\"ale
				\item Anzahl Kommunikationskan\"ale auf ein f\"ur das Projekt und das Team angemessenes Niveau halten
				\item regelm\"a\ss ige Treffen nicht als obligatorisch, sondern als verpflichtend einstufen
				\item jedem Treffen einer zuvor festgelegtem Thematik unterstellen
				\item Treffen moderieren (nicht im Sande verlaufen lassen und nicht in Details verlieren)
				\item Probleme direkt und m\"oglichst fr\"uh ansprechen, auch mit den Projektleitern
			\end{itemize}
		Alternativen: 
			\begin{itemize}
				\item Vertreter wechseln
				\item Kommunikationsprotokoll wird im Notfall von einem Team entwickelt und von den anderen Teams \"ubernommen
				\item Bereitstellen der Fahrzeuge f\"ur andere Teams, wenn diese ein Szenario testen wollen aber nicht jedes Team einen Vertreter schicken kann
			\end{itemize}
		
	\subsection{Personeller Ausfall}
		\textbf{Beschreibung}: Einzelne Mitglieder eines Teams tragen unmerklich oder nichts dem Projekt bei. Sie sind selten anwesend oder haben sich bereits vom Projekt verabschiedet. Die Gr\"unde sind vielf\"altig:
			\begin{itemize}
				\item \"Uberforderung (durch Projekt und/oder restliche Module im aktuellen Semester)
				\item organisatorische M\"angel
				\item Unzufriedenheit, Frust mit dem eigenen Team
				\item Gef\"uhl, man kommt nicht voran
				\item Krankheitb
			\end{itemize}
		\textbf{Auswirkung}: Der Workload der verbliebenen Teammitglieder verdichtet sich. \\
		\textbf{Tragweite}: projektgef\"ahrdend.\\
		Pr\"aventionsma\ss nahmen:
			\begin{itemize}
				\item angemessene Kommunikationskan\"ale aufbauen
				\item \"Uberforderung einzelner bereits im Vorfeld als Team vermeiden (Observierung des Workloads durch andere Teammitglieder)
				\item Aufgaben nach W\"unschen und Bef\"ahigung der Teammitglieder zuweisen
				\item regelm\"a\ss ige Motivationen (z.B. bei Erreichen von Meilensteinen eine kleine Feierlichkeit veranstalten)
				\item Ziele klar definieren, um Fortschritt auch sichtbar zu machen
				\item bei Scheitern einzelner Aufgaben als Team zur Probleml\"osung beitragen
				\item Team wechseln
			\end{itemize}
		Alternativen:
			\begin{itemize}
				\item 'Nachr\"ucker' in die Projektarbeit miteinbeziehen
				\item Teams zusammenlegen
				\item Arbeitsteilung unter den Teams
			\end{itemize} 
		
	\subsection{Workload durch Einarbeiten in Tools/Umgebungen}
		\textbf{Beschreibung}: Im Rahmen des Projekts sollen die Teams das Betriebssystem ROS 2 nutzen, um die Fahrzeuge zu steuern. Es erfodert eine Einarbeitung in den Umgang mit ROS 2.
		\textbf{Auswirkung}: Verz\"ogerung des eigentlichen Projekts\\
		\textbf{Tragweite}: kritisch, da nicht unmittelbar absehbar, wann am eigentlichen Projekt gearbeitet werden kann\\
		Pr\"aventionsma\ss nahmen:
			\begin{itemize}
				\item Einzelne arbeiten sich in die Funktionsweise von ROS 2 ein und schulen die restlichen Teammitglieder
				\item die Projektleitung unterweist die Teams in einem gemeinsamen Tutorium
				\item die Projektleitung stellt eine umfangreiche Dokumentation zu ROS 2 zur Verf\"ugung
		\end{itemize}
		Alternativen:
			\begin{itemize}
				\item Verwendung von ROS 1, da besser dokumentiert
				\item Kommunikation der Fahrzeuge und Verarbeiten der Signale mit eignen L\"osungen (Sockets etc.)
			\end{itemize}
		
	\subsection{Falsche Einsch\"atzung des Arbeitsaufwandes}
		\textbf{Beschreibung}: Einzelne Aufgaben wurden von ihrer zeitlichen Bearbeitung her falsch eingesch\"atzt.\\
		\textbf{Auswirkung}: Die Arbeitsabl\"aufe im Team m\"ussen ggf. neu organisiert werden. Die Arbeitsabl\"aufe der andere Teammitglieder ziehen sich in die L\"ange, falls ihre Aufgabe abh\"angig ist von der Aufgabe eines anderen Teammitglieds.\\
		\textbf{Tragweite}: in Teilbereichen kritisch, da bestimmte Aufgaben von der Erledigung anderer Aufgaben abh\"angig sind.\\
		Pr\"aventionsma\ss nahmen:
			\begin{itemize}
				\item Erkenntnisse ziehen aus dem (Nicht-)Einhalten von fr\"uheren Meilensteinen
				\item Erkenntnisse der anderen Teams nutzen
				\item Aufgabenstellung klar definieren
				\item Aufgaben m\"oglichst nach Pr\"aferenz vergeben
				\item im Team kritisch \"uberpr\"ufen, ob Priorit\"aten korrekt gesetzt wurden
			\end{itemize}
		Alternativen: 
			\begin{itemize}
				\item Aufgaben parat haben, welche nicht unmittelbar von der Erledigung einer anderen Aufgabe abh\"angig sind, um so effizient L\"ucken zu f\"ullen
				\item Sofern m\"oglich: Zusammenarbeit der Teammitglieder an der verz\"ogerten Aufgabe
			\end{itemize}
		
	\subsection{Abweichende Zielvorstellung des Kunden bzw. Auftragnehmers}
		\textbf{Beschreibung}: \"Uber die vereinbarten Akzeptanzkriterien herrscht zwischen Kunde und Auftragnehmer Dissens. Es ist ggf. nicht ersichtlich, auf welche Punkte man sich urspr\"unglich geeinigt hatte.\\
		\textbf{Auswirkung}: Unzufriedener Kunde.\\
		\textbf{Tragweite}: projektgef\"ahrdend.\\
		Pr\"aventionsma\ss nahmen:
			\begin{itemize}
				\item Erstellen eines Vertrages \"uber eindeutige Akzeptanzkriterien
				\item Gewissenhafte Auseinandersetzung mit den geforderten Akzeptanzkriterien
				\item F\"ur den Streitfall: Festhalten der Arbeitsschritte in einer umfassenden Dokumentation, um ggf. mit zuvor angefertigtem Vertrag abzugleichen
				\item Regelm\"a\ss ige Kommunikation mit dem Kunden, nicht nur zu Beginn des Projekts (konkret: Projekttreffen am Montag konsequent besuchen)
				\item Probleme m\"oglichst fr\"uh mit Kunden kl\"aren, ggf. Alternativen f\"ur Akzeptanzkriterien anbieten
			\end{itemize}
		Alternativen: /

\

\section{Quellen (Platzhalter)}


\end{document}

\subsection{}
	\textbf{Beschreibung}: \\
	\textbf{Auswirkung}: \\
	\textbf{Tragweite}: \\
	\textbf{Prävention}: \\
	\begin{itemize}
		\item
	\end{itemize}
	\textbf{Alternativen}: 

	[h4] Risiko Ausbaustufen
Risiko ABS 2 Obermenge DEMO
Risiko ABS 3 Obermenge DEMO
* Bodenbeschaffenheit verändert Kreisfahrverhalten
  * Risikominimierung:
    * Generalprobe vor Ort der DEMO
* Protokoll gibt Kreisfahrverhalten vor, wird von Fahrzeugen unterschiedlich ausgeführt
  * Risikominimierung:
    * Jedes Fahrzeug ist selbst verantwortlich Kreisfahrverhalten
* Kamera erkennt Hindernis vor sich aber ausserhalb des Kreises
  * Risikominimierung:
    * Kreis testen mit Hindernissen an den Seiten
* Akku fällt während fahrt aus
  * Risikominimierung:
    * Akkustatus auf Steuerrechner
    * Notaus Testfall "Auto tot"
* Auto verlässt Kreis unerwartet ( Szenario Kindertritt )
  * Risikominimierung:
    * Notaus, Testfall
* Hindernis taucht innerhalb des Bremsweges auf (Szenario Kindfall )
  * Risikominimierung:
    * Notaus, Testfall
* Hitzeentwicklung führt zu Fahrzeugausfall
  * Risikominimierung:
    * Temperatur als info im Steuercomputer
    * notaus bei spezifizierter temperaturschwelle
    * test bis an temperaturschwelle
Risiko ABS 4 Obermenge DEMO